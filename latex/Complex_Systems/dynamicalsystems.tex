% chaotic dynamics
A dynamical system is given by a space $M \subseteq R^d$ endowed with a collection of maps $\phi^t:M\rightarrow M$, where $t \in R$ and where $M$ is the phase space, with the "group property":
$$
	\phi^{t+s} = \phi^t o \phi^s
$$
The maps $\phi^t$ are usually called flows. \\ 
This formalization works for systems whose equations don't depend explicitly on time. \\
A typical example of a dynamical system is that of the solutions of a differential equation. So if we have a differential equation
$$
	\dot{x} = f(x)
$$
with initial condition $x(0) = x_0$, the flow is the function that takes the initial condition and returns the correct solution for the differential equation. So if this ODE has a global solution, so a solution $x(t)$ that satisfies the initial condition, then 
$$
	\phi^t(x_0) = x(t)
$$
We must also assume the uniqueness of the solution, that is, for one initial condition we only have one solution. //
Non-autonomous ODEs, where the equation field depends on time
$$
	\dot{x} = f(x,t)
$$
can be converted into autonomous ODEs in higher dimensions, going back to the previous simpler case. \\
We consider $\overline{M} = M \times R$ with $y = (x,t) \in \overline{M}$, and we define
$$
	F(x,t) = (f(x,t),1)
$$
with $f : M \times R \rightarrow R^d$ and $F : M \times R = \overline{M} \rightarrow R^{d+1}$. At this point we can write the new differential equation:
$$
	\dot{y} = F(y)
$$
with initial condition $y(0) = (x_0,0)$. For this to be useful, we must show that knowing the solution to this new equation implies knowing the solution for the original one. \\
Say $y(t) = (x(s),t(s))$ is a solution for the new equation
$$
	\begin{patrix}
		\dot{x}(s) \\ 
		\dot{t}(s)
	\end{patrix} = 
	F
	\begin{patrix}
		x(s) \\ 
		t(s)
	\end{patrix} =
	\begin{patrix}
		f(x(s),t(s)) \\ 
		1
	\end{patrix} = 
$$
$$
	\dot{x} = f(x,s) \ \ \ \ \ \dot{t} = 1
$$
with initial conditions $x(0) = x_0$ and $t(0) = 0$. For what concerns t, it's the identity function, so $t(s) = s$, which means that I can invert $t$ and $s$, so going back to the first equation we have
$$
	\dot{x} = f(x,t)
$$
which was our original equation. This means that a solution for the non autonomous equation is also a solution to the new autonomous one. \\
So, if we have a theory that treats non autonomous systems, that theory must also be able to treat autonomous systems. \\ \\
If the $t$ variable is a natural number, we call it $n$
$$
	\phi^n = T^n
$$

% dynamical systems with discrete time
% malthus
$$
	x_{n+1} = \alpha x_n
$$
where $x_n$ represents the population in year $n$. For $\alpha > 1$, we have population growth and for $\alpha < 1$ the population decreases. \\
If at time $t=0$ the population is $x_0$, at a given time we have
$$
	x_n = T_n(x_0) = \alpha^n x_0
$$
% velhurst
This model takes into account the fact that the higher the population is, the more resouces are needed, so there should be a limit that can't be surpassed, because the enviroment wouldn't be able to sustain it
$$
	x_{n+1} = \alpha x_n(A-x_n)
$$
Now we change variable, defining $\hat{x} = x/A$, and we get
$$
	\frac{x_{n+1}}{A} = A\alpha \frac{x_n}{A}\left(1 - \frac{x_n}{A}\right)
$$
$$
	\hat{x_{n+1}} = r \hat{x_n}(1-\hat{x_n})
$$
which is the logistic map:
$$
	T(x) = rx(1-x)
$$
We take $x$ between $0$ and $1$ and $r$ between $0$ and $4$. The graph is a parabola upside down.
