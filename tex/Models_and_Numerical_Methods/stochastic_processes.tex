\begin{definition}
    A \textbf{stochastic process} is an infinite sequence of random variables $X_n$ with values in $\mathcal{A}$ defined by the $k^\text{th}$ order joint distribution:
    \begin{equation*}
        \mu_k\left(a_1^k\right) = \mathbb{P}\left(X_1^k = a_1^k\right) \ \ a_1^k \in \mathcal{A}
    \end{equation*}
\end{definition}
We need also a consistency condition:
\begin{equation*}
    \mu_n\left(\omega_1^n\right) = \sum_{\omega_0 \in \mathcal{A}} \mu_{n+1}\left(\omega_0^n\right) = \sum_{\omega_{n+1} \in \mathcal{A}} \mu_{n+1}\left(\omega_1^{n+1}\right)
\end{equation*}
Equivalently, we can define a stochastic process through the conditional probability
\begin{equation*}
    \mu\left(\omega_n \vert \omega_1^{n-1}\right) = \frac{\mu_n\left(\omega_1^n\right)}{\mu_{n-1}\left(\omega_1^{n-1}\right)}
\end{equation*}
\begin{definition}
    A stochastic process is \textbf{stationary} if
    \begin{equation*}
        \mu\left(a_1^k\right) = \mu\left(a_{n+1}^{n+k}\right) \ \ \forall a_1^\infty \in \mathcal{A}^\mathbb{N}
    \end{equation*}
\end{definition}
\begin{definition}
    An \textbf{information source} is a stationary, ergodic, stochastic process.
\end{definition}
\begin{definition}
    A process or a source is a \textbf{shift-invariant Borel probability measure} $\mu$ on the topological space $\mathcal{A}^\mathbb{Z}$ of doubly-infinite sequences $x = \left\{x_n\right\}_{n \in \mathbb{Z}}$, drawn from a finite (i.e. countable) alphabet $\mathcal{A}$
\end{definition}
\begin{theorem}[Kolmogorov representation theorem]
    If $\left\{\mu_n\right\}$ is a sequence of measure defining a process then there is a unique Borel probability measure $\mu$ on $\mathcal{A}^\infty$ such that, $\forall k \geq 1$ and $\forall \left[a_1^k\right]$ cylinder
    \begin{equation*}
        \mu\left(\left[a_1^k\right]\right) = \mu_k\left(a_1^k\right)
    \end{equation*}
\end{theorem}